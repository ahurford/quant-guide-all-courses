\documentclass[]{book}
\usepackage{lmodern}
\usepackage{amssymb,amsmath}
\usepackage{ifxetex,ifluatex}
\usepackage{fixltx2e} % provides \textsubscript
\ifnum 0\ifxetex 1\fi\ifluatex 1\fi=0 % if pdftex
  \usepackage[T1]{fontenc}
  \usepackage[utf8]{inputenc}
\else % if luatex or xelatex
  \ifxetex
    \usepackage{mathspec}
  \else
    \usepackage{fontspec}
  \fi
  \defaultfontfeatures{Ligatures=TeX,Scale=MatchLowercase}
\fi
% use upquote if available, for straight quotes in verbatim environments
\IfFileExists{upquote.sty}{\usepackage{upquote}}{}
% use microtype if available
\IfFileExists{microtype.sty}{%
\usepackage{microtype}
\UseMicrotypeSet[protrusion]{basicmath} % disable protrusion for tt fonts
}{}
\usepackage{hyperref}
\hypersetup{unicode=true,
            pdftitle={Introduction to data entry, visualization, and statistics},
            pdfauthor={Amy Hurford},
            pdfborder={0 0 0},
            breaklinks=true}
\urlstyle{same}  % don't use monospace font for urls
\usepackage{natbib}
\bibliographystyle{apalike}
\usepackage{longtable,booktabs}
\usepackage{graphicx,grffile}
\makeatletter
\def\maxwidth{\ifdim\Gin@nat@width>\linewidth\linewidth\else\Gin@nat@width\fi}
\def\maxheight{\ifdim\Gin@nat@height>\textheight\textheight\else\Gin@nat@height\fi}
\makeatother
% Scale images if necessary, so that they will not overflow the page
% margins by default, and it is still possible to overwrite the defaults
% using explicit options in \includegraphics[width, height, ...]{}
\setkeys{Gin}{width=\maxwidth,height=\maxheight,keepaspectratio}
\IfFileExists{parskip.sty}{%
\usepackage{parskip}
}{% else
\setlength{\parindent}{0pt}
\setlength{\parskip}{6pt plus 2pt minus 1pt}
}
\setlength{\emergencystretch}{3em}  % prevent overfull lines
\providecommand{\tightlist}{%
  \setlength{\itemsep}{0pt}\setlength{\parskip}{0pt}}
\setcounter{secnumdepth}{5}
% Redefines (sub)paragraphs to behave more like sections
\ifx\paragraph\undefined\else
\let\oldparagraph\paragraph
\renewcommand{\paragraph}[1]{\oldparagraph{#1}\mbox{}}
\fi
\ifx\subparagraph\undefined\else
\let\oldsubparagraph\subparagraph
\renewcommand{\subparagraph}[1]{\oldsubparagraph{#1}\mbox{}}
\fi

%%% Use protect on footnotes to avoid problems with footnotes in titles
\let\rmarkdownfootnote\footnote%
\def\footnote{\protect\rmarkdownfootnote}

%%% Change title format to be more compact
\usepackage{titling}

% Create subtitle command for use in maketitle
\providecommand{\subtitle}[1]{
  \posttitle{
    \begin{center}\large#1\end{center}
    }
}

\setlength{\droptitle}{-2em}

  \title{Introduction to data entry, visualization, and statistics}
    \pretitle{\vspace{\droptitle}\centering\huge}
  \posttitle{\par}
  \subtitle{for first year biology}
  \author{Amy Hurford}
    \preauthor{\centering\large\emph}
  \postauthor{\par}
      \predate{\centering\large\emph}
  \postdate{\par}
    \date{2019-12-31}

\usepackage{booktabs}
\usepackage{amsthm}
\makeatletter
\def\thm@space@setup{%
  \thm@preskip=8pt plus 2pt minus 4pt
  \thm@postskip=\thm@preskip
}
\makeatother

\begin{document}
\maketitle

{
\setcounter{tocdepth}{1}
\tableofcontents
}
\chapter{Introduction}\label{introduction}

\begin{itemize}
\tightlist
\item
  Why is this important
\item
  Why have we made the choices we did (R, pedagogy citations)
\end{itemize}

\chapter{Entering data}\label{entering-data}

\chapter{Submitting your data to a
repository}\label{submitting-your-data-to-a-repository}

\chapter{How to install R and R
Studio}\label{how-to-install-r-and-r-studio}

\section{Why R}\label{why-r}

Biologists use a variety of tools and in recent years, the use of
computers has become widespread. Many different softwares are used by
biologists including Microsoft Office, R, Python, and ArcGIS. First year
biology, will introduce you to using R to visualize data, although we
note that many of the same end results could be generated using
different software. Our choice of R is for the following reasons:

\begin{enumerate}
\def\labelenumi{\arabic{enumi}.}
\item
  R programming is a valued skill: citation.
\item
  Reproducibility:
\item
  Accessibility: \texttt{R} is free.
\item
  No limits: while it is possible to need data visualization options or
  statistical analysis that are unavailable in \texttt{Microsoft\ Excel}
  this is rarely a problem in \texttt{R}: there is an \texttt{R} package
  for anything, from
  \href{https://rviews.rstudio.com/2017/10/09/population-modeling-in-r/}{serious}
  to
  \href{https://www.r-bloggers.com/useless-but-fun-r-packages/}{fun}).
\end{enumerate}

\section{R}\label{r}

\texttt{R} can be installed from \url{https://www.r-project.org/}. If
you have problems with installing \texttt{R} this same website provides
advice on how to seek support.

\section{RStudio}\label{rstudio}

\texttt{R} is a scripting language. \texttt{R\ Studio} is a graphical
user interface (GUI) that facilitates \texttt{R} coding by providing
buttons and menus to provide options for some commonly used commands.

\chapter{Finding your way around
RStudio}\label{finding-your-way-around-rstudio}

\chapter{Installing a package}\label{installing-a-package}

\chapter{A simple approach for loading data into
RStudio}\label{a-simple-approach-for-loading-data-into-rstudio}

\chapter{Making a graph with ggplot in
RStudio}\label{making-a-graph-with-ggplot-in-rstudio}

\chapter{Data submission}\label{data-submission}

\begin{itemize}
\tightlist
\item
  via upload file into github
\item
  direct edit of github
\end{itemize}

\bibliography{book.bib,packages.bib}


\end{document}
